%%%%%%%%%%%%%%%%%%%%%%%%%%%%%%%%%%%%%%%%%
%
% This file was adapted from the following original:
%
% Wilson Resume/CV
% Structure Specification File
% Version 1.0 (22/1/2015)
%
% This file has been downloaded from:
% http://www.LaTeXTemplates.com
%
% License:
% CC BY-NC-SA 3.0 (http://creativecommons.org/licenses/by-nc-sa/3.0/)
%
%%%%%%%%%%%%%%%%%%%%%%%%%%%%%%%%%%%%%%%%%

%----------------------------------------------------------------------------------------
%	PACKAGES AND OTHER DOCUMENT CONFIGURATIONS
%----------------------------------------------------------------------------------------

\usepackage{hyperref}

\usepackage[a4paper, hmargin=25mm, vmargin=30mm, top=20mm]{geometry} % Use A4 paper and set margins

\usepackage{fancyhdr} % Customize the header and footer

\setcounter{secnumdepth}{0} % Suppress section numbering

\usepackage[T1]{fontenc} % Output font encoding for international characters

\usepackage{fontspec} % Required for specification of custom fonts
\setmainfont[Path = ./fonts/,
Extension = .otf,
BoldFont = Erewhon-Bold,
ItalicFont = Erewhon-Italic,
BoldItalicFont = Erewhon-BoldItalic,
SmallCapsFeatures = {Letters = SmallCaps}
]{Erewhon-Regular}

\usepackage{sectsty} % Allows customization of titles

\fancypagestyle{plain}{\fancyhf{}\cfoot{\thepage\ of \pageref{LastPage}}} % Define a custom page style
\pagestyle{plain} % Use the custom page style through the document
\usepackage{nopageno} % no page numbers
\renewcommand{\headrulewidth}{0pt} % Disable the default header rule
\renewcommand{\footrulewidth}{0pt} % Disable the default footer rule

\usepackage[none]{hyphenat}

\setlength\parindent{0pt} % Stop paragraph indentation

% Non-indenting itemize
\newenvironment{itemize-noindent}
{\setlength{\leftmargini}{0em}\begin{itemize}}
{\end{itemize}}

% Text width for tabbing environments
\newlength{\smallertextwidth}
\setlength{\smallertextwidth}{\textwidth}
\addtolength{\smallertextwidth}{-2cm}

\usepackage{enumitem}
\newcommand{\sqbullet}{~\vrule height 1ex width .8ex depth -.2ex} % Custom square bullet point definition


\usepackage{tabularx}
%----------------------------------------------------------------------------------------
%	MAIN HEADER COMMAND
%----------------------------------------------------------------------------------------

%\renewcommand{\title}[1]{
%{\huge{\color{slateblue}\textbf{#1}}}\\ % Header section name and color
%\rule{\textwidth}{0.5mm}\\ % Rule under the header
%}
\renewcommand{\title}[1]{
{\huge{\textbf{#1}}}\\ % Header section name and color
\rule{\textwidth}{0.5mm}\\ % Rule under the header
}

%----------------------------------------------------------------------------------------
%	JOB COMMAND
%----------------------------------------------------------------------------------------

\newcommand{\job}[5]{
\begin{tabbing}
\hspace{2cm} \= \kill
\textbf{#1} \> {#3} \\
\textbf{#2} \>\+ \textit{#4} \\
\begin{minipage}{\smallertextwidth}
#5
\end{minipage}
\end{tabbing}
}
\newcommand{\jobheader}[4]{
\begin{tabbing}
\hspace{2cm} \= \kill
\textbf{#1} \> {#3} \\
\textbf{#2} \>\+ \textit{#4} \\
\end{tabbing}
}


%----------------------------------------------------------------------------------------
%	MY TABBED BLOCK COMMAND
%----------------------------------------------------------------------------------------
\newcommand{\mytabbedblock}[2]{
\begin{tabbing}
\hspace{2.2cm} \= \kill
\textbf{#1} \>
\begin{minipage}{\smallertextwidth}
#2
\end{minipage}
\end{tabbing}
}


%----------------------------------------------------------------------------------------
%	SKILL GROUP COMMAND
%----------------------------------------------------------------------------------------
\newcommand{\skills}[1]{
\begin{center}
\begin{tabular}{r|l}
\hline
#1
\end{tabular}
\end{center}
}

\newcommand{\skillgroup}[2]{
\textbf{#1} & #2 \\
\hline
}

%----------------------------------------------------------------------------------------
%	TABBED BLOCK COMMAND
%----------------------------------------------------------------------------------------

\newcommand{\tabbedblock}[1]{
\begin{tabbing}
\hspace{2cm} \= \hspace{4cm} \= \kill
#1
\end{tabbing}
}
